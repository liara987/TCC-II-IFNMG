% -----------------------------------
% METODOLOGIA DA PESQUISA
% -----------------------------------
\chaptertitlename{Metodologia}
\begin{flushleft}
	A metodologia de desenvolvimento deste trabalho consistiu em cinco etapas:
\end{flushleft}

\begin{enumerate}
		\item Compreensão da estrutura da plataforma web;
		\item Elaboração de um protótipo de aplicativo para aparelhos móveis adequado às características da plataforma;
		\item Definição dos pré-requisitos para viabilizar a conexão entre a plataforma web e o aplicativo;
		\item Implementação do aplicativo para dispositivos móveis na plataforma livre Android;
		\item Alimentação da plataforma;
\end{enumerate}

A principio foi realizado uma análise da plataforma Cidadão do Vale, para compreensão do funcionamento e estruturação do sistema. Foi considerado que a implementação tem sido realizada de maneira coletiva e colaborativa, e que se trata de um sistema já em funcionamento, a familiarização com a interface e programação da plataforma foi uma etapa crucial para o desenvolvimento.

A segunda fase, elaborou-se um protótipo do aplicativo no qual buscou replicar os elementos da interface da plataforma web de maneira intuitiva e amigável.

No terceiro momento, foi estudado os pré-requisitos para viabilizar a conexão entre a plataforma web e o aplicativo. Neste caso, foram avaliadas as características e vantagens de se desenvolver aplicativos dos tipos nativos, Web Apps, ou híbridos. Segundo informações de \citeonline{madureira_aplicativo_2017}, em síntese, eles se diferenciam pela linguagem utilizada e pelo tráfego de dados utilizados. Os nativos são programados em linguagem exclusiva para dispositivos móveis, são mais rápidos e confiáveis. Os Web Apps não são aplicativos reais, mas sim sites desenvolvidos para dispositivos móveis. As vantagens são o funcionamento em qualquer sistema operacional e o fato de não ocupar espaço na memória dos aparelhos. A desvantagem é que requer conexão com a internet para funcionamento. Por fim, os aplicativos híbridos são aqueles em que mesclam características dos nativos e Web Apps. Sua elaboração é mais simples, entretanto, requer internet para o funcionamento e não possui a mesma velocidade de resposta que um nativo. 

Após análizar todas as possibilidade, optou-se por desenvolver o aplicativo de forma hibrida, utilizando o framework Cordova. A decisão foi tomada com base nas necessidades de realizar contribuições sem internet, ter uma boa integração com a página web e que usa-se uma linguagem padrão que converse com o site, além destes pontos o Cordova possui uma boa documentação, é de código aberto, oque da uma maior liberdade para adaptação e desenvolve aplicativos multiplataformas.

Durante todo o desenvolvimento foram realizados testes, tais testes permitiram aferir a qualidade do seu funcionamento, incluindo a velocidade e capacidade de transmissão de dados, intuitibilidade, manuseio, erros e travamentos. Está fase foi primordial para o desenvolvimento, e foi realizada constantemente durante o processo.

Por fim foi realizado a alimentação da plataforma com informações atuais sobre o município. O aplicativo será divulgado e disponibilizado para o público, para coleta e análise das informações atualizadas.

\chapter{Resultados obtidos}